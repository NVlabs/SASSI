\section{\hl{IMPORTANT: Required compiler flags}}
\label{sec:important-required}

This section discusses some flags and constraints that that SASSI
requires for proper operation.  While the complete examples that we
show in the next several sections will discuss these flags and
constraints, it is important to highlight them here, because \hl{
  failure to follow this recipe will cause your instrumented programs
  to fail:}

\begin{itemize}
  \item \hl{You must compile your instrumentation handlers using only
    16 registers:} GPGPU programs often using a staggering number of
    registers.  The SASSI approach of injecting function calls implies
    that live registers must be saved and restored across call sites.
    By forcing the instrumentation handlers to use only 16 registers,
    SASSI can drastically limit the number of saves and restores
    required across handler call boundaries.  SASSI instrumentation
    will only save and restore the first 16 live registers, so if your
    handler uses registers outside this set, there's a good chance
    your instrumented program will break.  \hl{This restriction only
      applies to the instrumentation handlers, not the programs you
      instrument-- they can still use as many registers as the user deems
      necessary.}  We can trivially set the number of registers to 16
    with the following \texttt{nvcc} flag:
    \texttt{--maxrregcount=16}.
  \item \hl{You must explicitly specify what \emph{code} you want to
    generate:} The CUDA toolchain allows users to generate two forms
    of code: virtual and real ISA code.  In the case that you generate
    virtual ISA code (e.g., \texttt{compute\_35}, \texttt{compute\_50}), the display
    driver will just-in-time compile your code when it is loaded.
    This will not work for SASSI because SASSI is not in the driver's
    compiler.  \hl{Instead, you must specify a real ISA, which always
    starts with an ``sm'' (e.g., \texttt{sm\_35}, \texttt{sm\_50}).  This will cause the
    toolchain to precompile your code with \texttt{ptxas}.}  For
    example, if we are targeting an \texttt{sm\_50}, I would probably
    supply the following option to \texttt{nvcc} to generate the
    correct code: \texttt{-gencode arch=compute\_50,}\hl{\texttt{code=sm\_50}}.
    Please run \texttt{nvcc -h} for details about this option.
  \item \hl{You must generate position independent code with
    \texttt{-dc} (or equivalently, \texttt{-rcd=true}):} SASSI
    instrumentation handlers are linked in with an instrumented
    application.  To make ``cross-module'' calls with the CUDA
    toolchain, you must use this flag.  While position independent
    code adds modularity to GPU programs and brings CUDA more in line
    with general purpose computing, it can affect the performance of
    the generated code, sometimes significantly.  Future work will try
    to remove this requirement.
\end{itemize}
